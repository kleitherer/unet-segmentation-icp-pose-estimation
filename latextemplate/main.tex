\documentclass[12pt,letterpaper]{article}
\usepackage{amsmath}
\usepackage{amsthm}
\usepackage{amsfonts}
\usepackage{amssymb}
\usepackage{amscd}
\usepackage{enumerate}
\usepackage{fancyhdr}
\usepackage{mathrsfs}
\usepackage{bbm}
\usepackage{framed}
\usepackage{mdframed}
\usepackage{listings}
\usepackage{cancel}
\usepackage{mathtools}
\usepackage{verbatim}
\usepackage{enumitem}
\usepackage{makecell}
\usepackage{soul}
\usepackage[letterpaper,voffset=-.5in,bmargin=3cm,footskip=1cm]{geometry}
\usepackage[colorlinks = true]{hyperref} 
\setlength{\parindent}{0.0in}
\setlength{\parskip}{0.1in}
\allowdisplaybreaks
\headheight 15pt
\headsep 10pt
\newcommand\N{\mathbb N}
\newcommand\Z{\mathbb Z}
\newcommand\R{\mathbb R}
\newcommand\Q{\mathbb Q}
\newcommand\lcm{\operatorname{lcm}}
\newcommand\setbuilder[2]{\ensuremath{\left\{#1\;\middle|\;#2\right\}}}
\newcommand\E{\operatorname{E}}
\newcommand\V{\operatorname{V}}
\newcommand\Pow{\ensuremath{\operatorname{\mathcal{P}}}}

\DeclarePairedDelimiter\ceil{\lceil}{\rceil}
\DeclarePairedDelimiter\floor{\lfloor}{\rfloor}
\newcommand\hint[1]{\textbf{Hint}: #1}
\newcommand\note[1]{\textbf{Note}: #1}

\lstset{
  basicstyle=\ttfamily,
  columns=fullflexible,
  frame=single,
  breaklines=true,
  postbreak=\mbox{\textcolor{red}{$\hookrightarrow$}},
}

\fancypagestyle{firstpagestyle} {
  \renewcommand{\headrulewidth}{0pt}
  \lhead{\textbf{Robot Perception}}
  \chead{\textbf{}}
  \rhead{Autumn 25-26}
}

\pagestyle{fancyplain}
\usepackage{tikz}

\begin{document}
  \thispagestyle{firstpagestyle}
  \begin{center}
    {\huge \textbf{Homework 3}}
  \end{center}


 
\subsection*{Problem 1 (10 pts): }

\subsubsection*{1.1 Convolution -- 10 points}


Consider a 4 $\times$ 4 grayscale image represented by the following matrix:
\[ \begin{bmatrix}
2 & 1 & 3 & 0 \\
0 & 1 & 2 & 3 \\
3 & 0 & 2 & 1 \\
1 & 2 & 0 & 3 \\
\end{bmatrix} \]
Define a \(2 \times 2\) kernel matrix for a convolution operation:
\[ \begin{bmatrix}
-1 & 1 \\
2 & 0 \\
\end{bmatrix} \]
Perform a convolution operation on the given image using this kernel (stride=1, padding=0). Show the resulting feature map (output matrix). 

\textcolor{blue}{Answer:}

\subsection*{Problem 2: Coding Assignments}
\subsubsection*{2.1.3}
\textcolor{blue}{Answer:}

\subsubsection*{Learning Curve}
\textcolor{blue}{Insert learning curve in this subsection}
\subsubsection*{Point cloud visualization}
\textcolor{blue}{Insert point cloud visualization in this subsection}




\end{document}